\documentclass{article}
\usepackage{amsmath}
\usepackage{fullpage}
\usepackage{logicproof}
\usepackage{parskip}
\usepackage{forest}
\usepackage{float}
\usepackage{hyperref}

\edef\restoreparindent{\parindent=\the\parindent\relax}
\usepackage{parskip}
\restoreparindent

\title{CMPE 58S: Sp. Tp. Computer Aided Verification \\ Homework III: DPLL}
\date{2018 October 22}
\author{Mehmet Utkan Gezer \\ 2018700060}

\begin{document}
\maketitle

This homework is about making a SAT solver of our own.
We are expected to implement a DPLL SAT solver that is
written in C/C++ or Java. To be fast, it should incorporate
heuristics such as backtracking, unit propagation and
pure literal elimination. We are, however, forbidden to
implement other heuristics, such as CDCL or
conflict-directed backjumping.

\section{Implementing the DPLL}

With the speed stressed in the description, I have decided
to implement my sat solver in C. even when I disregard the
speed requirement, I would prefer C over C++ or Java,
although I would much rather use a higher level language
if it was allowed. For an assignment to teach the basics,
I believe we should be able to deal with this task on a
superficial level with a higher level language like
Python or Julia, and concern ourselves less with optimizing
the structures we use for the heuristics we'll be employing.

My implementation of the SAT solver is, again, in C, and
is bundled with this document under the name
\texttt{satsolver.c}. We have compiled it both on
Windows (MinGW/gcc) and Ubuntu (gcc), with the
\texttt{-Wall} flag enabled, and it should compile
without warnings.

We have tested our implementation against memory
leaks using \texttt{valgrind}, and it should have no
memory leaks.

\section{Satisfying solutions to 5 problems}

Our implementation fails to provide a satisfying
solution to the 4th problem under a reasonable amount
of time, so we omit the output for it. While it was,
at some point, able to give a satisfying assignment
to the 5th problem, it cannot do so now, unfortunately.
It is possible that the desktop computer we were trying
with before was more capable, or some other reason
during the course of development to improve the
algorithm in speed and space efficiency.

For problems 1, 2 and 3, the solutions can be found
in files \texttt{problem1.sol}, \texttt{problem2.sol}
and \texttt{problem3.sol}, respectively.

\section{Determining satisfiability of 7 problems}

Following are the satisfiability status of the
corresponding problems, determined by our implementation.

\begin{center}
    \begin{tabular}{l l}
        \texttt{problem6.cnf}  & Satisfiable \\
        \texttt{problem7.cnf}  & Not Satisfiable \\
        \texttt{problem8.cnf}  & Satisfiable \\
        \texttt{problem9.cnf}  & Satisfiable \\
        \texttt{problem10.cnf} & Not Satisfiable \\
        \texttt{problem11.cnf} & Satisfiable \\
        \texttt{problem12.cnf} & Satisfiable
    \end{tabular}
\end{center}

\section{Discussion about problem size and time it takes}

As per request, here is the table of times it took for our
SAT solver implementation to solve the previous 7 problems.
We mention the problem characteristics in the table for
further discussion. Problem \#10 was solved on a desktop
computer with our algorithm before, but we could not
get it solved in under 15 minutes on a 3-years-old
tablet-like laptop, with which we are timing now.

\begin{center}
    \begin{tabular}{l || l | l | l | l || l}
    Problem\#  & Sat? & \#Vars & \#Clauses & \#C/\#V & Time (seconds) \\\hline
    6          & Yes  & 50     & 100       & 2       & immediate \\
    7          & No   & 50     & 100       & 2       & immediate \\
    8          & Yes  & 50     & 300       & 6       & immediate \\
    9          & Yes  & 100    & 200       & 2       & about 55 \\
    10         & No   & 100    & 200       & 2       & over 15 minutes \\
    11         & Yes  & 100    & 600       & 6       & immediate \\
    12         & Yes  & 200    & 1200      & 6       & about 25
    \end{tabular}
\end{center}

It is hard to find a correlation in this data,
as there are too many data points with the immediate
output.

It looks like the problem is solved rather quickly for;
\begin{itemize}
    \item Problems with a fewer variables.
    \item Problems with a higher clause-to-variable ratio.
\end{itemize}

It makes sense that fewer variables makes the problem
easier, as the search space is exponential with the
power of number of variables. It also makes sense
that increase in clause-to-variable ratio affects
runtimes positively, since more clauses increase the chance
for us to encounter more unit-clauses and empty-clauses
which are the boosting heuristics of our implementation
of DPLL.

Problems with many variables and too few clauses seem
to take the longest. Specifically when the problem
is unsatisfiable, the attempting to exhaust the search
space becomes much harder when there aren't enough
clauses to introduce unit-clauses and empty-clauses.

\section{Bonus: Comparison with Minisat}

Compared to the Minisat, our implementation suffers from
heavy memory usage and exessive computation time,
specifically with the 4th problem.

I had been discussing this homework with a friend of mine
studying SAT problems abroad in CMU, and just today
he got curious about this specific problem that my
implementation is struggling with. He said it could
be a ``Random SAT'', and then tested this using
his WalkSAT algorithm written in Python.

Since his Python implementation of WalkSAT was able to
solve the question in under 4 seconds, he was certain
that it is a Random SAT problem, and then informed me
that it is very much expected that a DPLL implementation
will suffer against it.

While I am sure I could have improved my implementation,
specifically at its memory consumption, this explanation
of his did bring me some relief.

\end{document}
